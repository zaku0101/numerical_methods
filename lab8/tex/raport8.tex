\documentclass{article}
\usepackage{listings}
\usepackage{color}
\usepackage{amsmath}
\usepackage{mathtools}
\usepackage{amsfonts}
\usepackage{amssymb}
\usepackage{caption}
\usepackage{tabularx}
\usepackage[export]{adjustbox}
\usepackage{polski}
\usepackage{indentfirst}
\usepackage{graphicx}
\usepackage{pdfpages}
\usepackage{float}
\usepackage{gauss}

\DeclareCaptionType{equ}[][List of equations]
\captionsetup[equ]{labelformat=empty}

%script adding bars in matrix
\usepackage{etoolbox}
\makeatletter
\patchcmd\g@matrix
 {\vbox\bgroup}
 {\vbox\bgroup\normalbaselines}% restore the standard baselineskip
 {}{}
\makeatother

\newcommand{\BAR}{%
  \hspace{-\arraycolsep}%
  \strut\vrule % the `\vrule` is as high and deep as a strut
  \hspace{-\arraycolsep}%
}
\definecolor{dkgreen}{rgb}{0,0.6,0}
\definecolor{gray}{rgb}{0.5,0.5,0.5}
\definecolor{mauve}{rgb}{0.58,0,0.82}

\lstset{frame=tb,
  language=Python,
  aboveskip=3mm,
  belowskip=3mm,
  showstringspaces=false,
  columns=flexible,
  basicstyle={\small\ttfamily},
  numbers=none,
  numberstyle=\tiny\color{gray},
  keywordstyle=\color{blue},
  commentstyle=\color{dkgreen},
  stringstyle=\color{mauve},
  breaklines=true,
  breakatwhitespace=true,
  tabsize=3,
  extendedchars=\true,
  inputencoding=utf8x
}

\lstset{literate={ą}{{\k{a}}}1 {ł}{{\l{}}}1 {ń}{{\'n}}1 {ę}{{\k{e}}}1 {ś}{{\'s}}1 {ż}{{\.z}}1 {ó}{{\'o}}1 {ź}{{\'z}}1 {Ą}{{\k{A}}}1 {Ł}{{\L{}}}1 {Ń}{{\'N}}1 {Ę}{{\k{E}}}1 {Ś}{{\'S}}1 {Ż}{{\.Z}}1 {Ó}{{\'O}}1 {Ź}{{\'Z}}1 }

\begin{document}
\title{Sprawozdanie - Metody numeryczne i optymailzacja}
\author{Jakub Andryszczak 259519,\\ Jakub Żak 244255,\\ Maciej Cierpisz 249163}
\date{}
\maketitle

\newpage
\tableofcontents
%Tutaj zaczyna się wstęp

\newpage
\section{Zadanie nr. 1}



\section{Zadanie nr. 2}

\section{Zadanie nr. 3}

Rozwiąż poniższy układ równań nieliniowych:\\
\begin{equation}
  \begin{cases}
    2x_1=x_2=exp(-x_1)\\
    -x_1+2x_2=exp(-x_2)
  \end{cases}
\end{equation}

dla punktu startowego $x_0=[-5 5]^t$.

Poniżej kod realizujący zadanie:

\begin{lstlisting}
  
  import numpy as np
  from scipy.optimize import fsolve
  
  # Define the system of nonlinear equations
  def equations(vars):
      x1, x2 = vars
      eq1 = 2 * x1 - x2 - np.exp(-x1)
      eq2 = -x1 + 2 * x2 - np.exp(-x2)
      return [eq1, eq2]
  
  # Initial guess
  x0 = [-5, -5]
  
  # Solve the system of equations
  solution = fsolve(equations, x0)
  
  # Print the solution
  print("Solution:")
  print(f"x1 = {solution[0]}")
  print(f"x2 = {solution[1]}")

\end{lstlisting}

Wyniki:

\begin{equation}
  \begin{cases}
    x_1=0.5671432904097838\\
    x_2=0.567143290409784
  \end{cases}
\end{equation}

\section{Zadanie nr. 4}
Znajdź rozwiązanie minimalizujące funkcję celu\\
$\sum_{k=1}^{10}(2+2k-exp(kx_1)-exp(kx_2))^2$\\
dla punktu początkowego $x_0=[0.3, 0.4]^T$.

Poniżej kod realizujący zadanie:

\begin{lstlisting}
  import numpy as np
from scipy.optimize import minimize

# Define the objective function
def objective(x):
    x1, x2 = x
    total = 0
    for k in range(1, 11):
        total += (2 + 2*k - np.exp(k*x1) - np.exp(k*x2))**2
    return total

# Initial guess
x0 = [0.3, 0.4]

# Minimize the objective function
result = minimize(objective, x0)

# Print the solution
print("Solution:")
print(f"x1 = {result.x[0]}")
print(f"x2 = {result.x[1]}")
\end{lstlisting}

Wyniki:

\begin{equation}
  \begin{cases}
    x_1=0.25782520984040996\\
    x_2=0.2578252098334402
  \end{cases}
\end{equation}

\section{Zadanie nr. 5}



\end{document}